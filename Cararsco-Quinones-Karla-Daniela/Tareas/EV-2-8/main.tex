\documentclass[letterpaper]{article}
\usepackage{underscore}
\usepackage[left=2.0cm, right=2.0cm, top=1.0cm]{geometry}
\usepackage[utf8]{inputenc}
\usepackage{graphicx}
\usepackage{graphics}
\usepackage[spanish]{babel}
\usepackage{lipsum}
\usepackage{float}
\usepackage{subfigure}
\usepackage{csquotes}
\usepackage{color}
\usepackage{colortbl}
\usepackage{xcolor}

\title{EV\_2\_8\_calcular\_los\_parametros\_de\_circuitos\_de\_activación\_de\_transistores\_de\_potencia}
\author{Carrasco Quiñones Karla Daniela}
\date{28/October/2019}

\begin{document}
\begin{figure}[t]
    \includegraphics[width=6cm]{img/logo.png}
\end{figure}
\vspace{2cm}
\maketitle
\vspace{12cm}
\begin{center}
   Universidad politécnica de la zona metropolitana de Guadalajara.\\
Sistemas electrónicos de interfaz.\\
4-A Mecatrónica.\\ 
\end{center}
\newpage
%\begin{LARGE}
%\textbf{Calculo resistencia base:}
%\end{LARGE}
\begin{large}
    Para la resistencia que se debe emplear en la base para que el trancistor actue en corte y saturación se hace la fórmula siguiente donde:\\\\
    $R_{base}$ es la resistecia que se empleará.\\\\
    $V_{IN}$: Es el voltaje que va a entrar en la base.\\\\
    $V_{Activacion}$: Es el voltaje de ruptura en el transistor\\\\
    $hFe$ = Forward current gain, es la ganancia que tendrá en corriente un transistor, este se obtiene al medir el transistor con el multímetro o mediante el data sheet.\\\\
    $I_{c}$ es el consumo del circuito conectado al transistor de colector a emisor.\\\\
    El valor de resistencia obtenido será el ideal para mantener el transistor en conducción sin dañar la base con demasiado flujo de corriente. 
    \vspace{.3cm}
    
    \begin{figure}[htbp]
    \begin{huge}
    $R_{base}=\frac{(V_{IN}-V_{Activacion})hFe)}{I_{c}}$
    \end{huge}
    \end{figure}
    
    \begin{itemize}
        \item Se busca el material de nuestro transistor
        \begin{itemize}
            \item Germanio = 0.3V
            \item Silicio = 0.7V
        \end{itemize}
        \item Para después identificar el voltaje que entrará, este llamado Voltaje base/colector
        \item Con el uso del multímetro se buscará el hFe del transistor seleccionado
        \item Para obtener el valor de $I_c$ se debe utilizar ley de Ohm, con la formula $I_c=\frac{V_{cc}}{R_{cc}}$
        \item Se sustituirán los valores dependiedndo lo calculado, para posteriormente reducir los terminos de arriba y ultimadametne dividir. Obteniendo así en resultado de activación del transistor
    \end{itemize}
    \begin{large}
        \textbf{Ejemplo:}\\\\
        Se tiene un circuito que va a entregar 5v a un transistor 2N2222 que activara un motor con un consumo de 400mA.\\
        ¿Cual es la resistencia que se requiere en la base del transistor?\\
        \begin{enumerate}
            \item Se identifican los componentes de la formula.\\\\
            $V_{IN}$ = 5v\\
            $V_{Activacion}$ = 0.7\\
            $hFe$ = 75\\
            $I_{c}$ = 400mA\\
            \item Reemplazamos en la formula quedando:\\\\
                $R_{base}= \frac{(5V-0.7V)*75}{0.4A}$\\
                $*$En la formula se utilizan amperes asi que 400mA = 0.4A.\\\\
            \item Resolucion de la formula:\\\\
                Se reduce hasta dejar una simple fracción\\
                $R_{base}= \frac{(4.3)*75}{0.4A}$\\\\
                Se resuelve la fracción.\\
                $R_{base}= \frac{322.5}{0.4A}$\\\\
                $R_{base}= 806.25\Omega$\\\\
       
       
        \end{enumerate}
    \end{large}
\end{large}
\begin{thebibliography}{}
\bibitem{formula}\textsc{Anonimo.} \textit{Calcular la resistencia para un transistor accionado por un microcontrolador.} Recuperdo el 29/10/2019 de:\\
https://www.sistemasorp.es/2011/10/05/calcular-la-resistencia-para-un-transistor-accionado-por-un-microcontrolador/\\
\bibitem{DataSheet}\textsc{PHILIPS.} \textit{2N2222;2N2222A NPN switching transistors.} Recuperdo el 29/10/2019 de:\\
https://pdf1.alldatasheet.com/datasheet-pdf/view/15067/PHILIPS/2N2222.html

\end{thebibliography}


\end{document}
